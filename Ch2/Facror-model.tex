\documentclass{article}

\usepackage{arxiv}

\usepackage[utf8]{inputenc} % allow utf-8 input
\usepackage[T1]{fontenc}    % use 8-bit T1 fonts
\usepackage{hyperref}       % hyperlinks
\usepackage{url}            % simple URL typesetting
\usepackage{booktabs}       % professional-quality tables
\usepackage{amsfonts}       % blackboard math symbols
\usepackage{nicefrac}       % compact symbols for 1/2, etc.
\usepackage{microtype}      % microtypography
\usepackage{lipsum}

\title{Factor model}

\author{
    Xie zejian
   \\
     \\
   \\
  \texttt{\href{mailto:xiezej@gmail.com}{\nolinkurl{xiezej@gmail.com}}} \\
  }

\newlength{\cslhangindent}
\setlength{\cslhangindent}{1.5em}
\newenvironment{cslreferences}%
  {\setlength{\parindent}{0pt}%
  \everypar{\setlength{\hangindent}{\cslhangindent}}\ignorespaces}%
  {\par}


\begin{document}
\maketitle

\def\tightlist{}


\begin{abstract}

\end{abstract}


\hypertarget{capm}{%
\subsection{CAPM}\label{capm}}

\hypertarget{beta-representation}{%
\subsubsection{Beta representation}\label{beta-representation}}

Recall the tangency portfolio is
\(\omega_D=\frac{\mathbf{V^-}(\overline{\mathbf{r}}-r_f\mathbf{e})}{\mathbf{e'V^-}(\overline{\mathbf{r}}-r_f\mathbf{e})}\).
Write \(\omega_D=m \mathbf{V^-} (\overline{\mathbf{r}}-r_f\mathbf{e})\)
where
\(m=\frac{1}{\mathbf{e'}\mathbf{V^-} (\overline{\mathbf{r}}-r_f\mathbf{e})}\),
then we have

\[ \overline{r}-r_f\mathbf{e}=\frac{1}{m}\mathbf{V}\omega_D \]

Note \(\cov(\mathbf{r},\omega'\mathbf{r})=\mathbf{V\omega}\) and

\[ \sigma_D^2=\omega_D'\mathbf{V}\omega_D=m\omega'_D(\overline{\mathbf{r}}-r_f\mathbf{e})= m\overline{r}_D-mr_f \]

we have

\[ \overline{\mathbf{r}}-r_f\mathbf{e}=\frac{\overline{r}_D-r_f}{\sigma_D^2}\cov(\mathbf{r},{r_D}) \]

Denote \(\frac{\cov(\mathbf{r},r_D)}{\sigma_D^2}=\beta_D\), we have

\[ \overline{\mathbf{r}}-r_f\mathbf{e}=\beta_D({\overline{r}_D-r_f}) \]

Similar results also holds for any portfolio \(\overline{r}_p\) and the
zero covariance portfolio \(\overline{r}_q\) in the MVF:

\[ \overline{\mathbf{r}}-\overline{r}_q\mathbf{e}=\beta_p(\overline{r}_p-\overline{r}_q) \]

It's clear in the view of every portfolio \(\overline{r}_p\) is also a
tangency portfolio by selecting proper \(r_f\). One can also check it in
a dirty way:

\textbf{Proof} Suppose \(r_p\) and \(r_q\) both in the MVF without
risk-free asset, recall

\[ \begin{aligned}
  \omega_p'\mathbf{V}\omega_q=\frac{1}{\delta}+\frac{\delta(\overline{r}_p-\frac{\alpha}{\delta})(\overline{r}_q-\frac{\alpha}{\delta})}{\delta\xi-\alpha^2}
\end{aligned} \]

If the covariance is \(0\), we have

\[ \overline{r}_q=\frac{\alpha}{\delta}-\frac{\delta\xi-\alpha^2}{\delta^2(\overline{r}_p-\alpha/\delta)} \]

Then

\[ \begin{aligned}
  \mathbf{\overline{r}}-\overline{r}_q \mathbf{e}&=\mathbf{\overline{r}}-(\frac{\alpha}{\delta}-\frac{\delta\xi-\alpha^2}{\delta^2(\overline{r}_p-\alpha/\delta)})\mathbf{e}
  \\&=\frac{1}{\delta^2(\overline{r}_p-\alpha/\delta)}(\delta^2(\overline{r}_p-\alpha/\delta))(\mathbf{\overline{r}}-(\frac{\alpha}{\delta}-\frac{\delta\xi-\alpha^2}{\delta^2(\overline{r}_p-\alpha/\delta)})\mathbf{e})
  \\&=\frac{1}{\delta^2(\overline{r}_p-\alpha/\delta)}(\mathbf{\overline{r}}(\delta^2(\overline{r}_p-\alpha/\delta))-(\alpha\delta(\overline{r}_p-\alpha/\delta)-(\delta\xi-\alpha^2))\mathbf{e})
  \\&=\frac{1}{\delta^2(\overline{r}_p-\alpha/\delta)}(\mathbf{\overline{r}}(\delta^2(\overline{r}_p-\alpha/\delta))-(\alpha\delta\overline{r}_p-\delta\xi)\mathbf{e}
  \\&=\frac{
    (\delta^2\overline{r}_p\mathbf{\overline{r}}-\alpha\delta\mathbf{\overline{r}})-(\alpha\delta\overline{r}_p-\delta\xi)\mathbf{e}}
  {\delta^2(\overline{r}_p-\alpha/\delta)}
  \\&=\frac{
    (\delta\overline{r}_p-\alpha)\mathbf{\overline{r}}-(\alpha\overline{r}_p-\xi)\mathbf{e}}
  {\delta(\overline{r}_p-\alpha/\delta)}
\end{aligned} \]

On the other hand:

\[ \begin{aligned}
  \beta_p&=\frac{\mathbf{V\omega_p}}{\omega_p'\mathbf{V}\omega_p}
  \\&=\frac{1}{\omega_p'\mathbf{V}\omega_p}(\lambda_p\overline{\mathbf{r}}+\gamma \mathbf{e})
  \\&=\frac{1}{\omega_p'\mathbf{V}\omega_p}(\frac{\mathbf{\xi e-\alpha  \overline{r}}}{\delta\xi-\alpha^2}+\frac{\mathbf{-\alpha e+\delta\overline{r}}}{\delta\xi-\alpha^2}\overline{r}_p)
  \\&=\frac{1}{\omega_p'\mathbf{V}\omega_p}(\frac{ (\delta\overline{r}_p-\alpha)\mathbf{\overline{r}}-(\alpha\overline{r}_p-\xi)\mathbf{e}}{\Delta})
\end{aligned} \]

Then it's remain to show that

\[ (\overline{r}_p-\overline{r}_q)\delta(\overline{r}_p-\alpha/\delta)=\omega'\mathbf{V}\omega\Delta \]

It's clear since

\[ \omega'\mathbf{V}\omega \Delta=\sigma_p^2\Delta=\frac{\Delta}{\delta}+\delta(\overline{r}_p-\frac{\alpha}{\delta})^2 \]

and

\[ \begin{aligned}
  (\overline{r}_p-\overline{r}_q)\delta(\overline{r}_p-\alpha/\delta)&=
  ((\overline{r}_p-\frac{\alpha}{\delta})+\frac{\delta\xi-\alpha^2}{\delta^2(\overline{r}_p-\alpha/\delta)})\delta(\overline{r}_p-\alpha/\delta)
  \\&=\frac{\Delta}{\delta}+\delta(\overline{r}_p-\frac{\alpha}{\delta})^2
\end{aligned} \]

\hypertarget{capm-1}{%
\subsubsection{CAPM}\label{capm-1}}

In capital market equilibrium, the market portfolio is tangecy portfolio
\(\overline{r}_D={r}_m\), then

\[ \bm{\mathbf{\overline{r}-\mathit{r_f}e}}=\bm{\mathbf{\beta_m}}(\overline{r}_m-r_f) \]

where

\[ \beta_m=\begin{bmatrix}
  \frac{\cov(r_1,\overline{r}_m)}{\sigma^2_m}\\
  \frac{\cov(r_2,\overline{r}_m)}{\sigma^2_m}\\
  \cdots\\
  \frac{\cov(r_n,\overline{r}_m)}{\sigma^2_m}\\
\end{bmatrix} \]

this equation is called \textbf{Sharpe-Lintner CAPM}.
\(\overline{r}_m-r_f\) is called \textbf{market risk premium} and
\(\frac{\overline{r}_m-r_f}{\sigma_m}\) is called \textbf{market sharpe
ratio}. Translate it from vector form, we get the \textbf{Security
Market Line}:

\[ 
r_i-r_f=\beta_{i,m}(\overline{r}_m-r_f) 
\]

\hypertarget{realized-return}{%
\subsubsection{Realized return}\label{realized-return}}

Now consider both \(r_i\) and \(r_m\) is random variable, let
\(\epsilon\) be a random vector with zero expection and zero covariance
with \(r_i\) and \(r_m\), then

\[ 
r_i-r_f=\beta_{i,m}(r_m-r_f)+\epsilon_i
\]

This is a regression equation, if one include an intercept, then the
model

\[ 
r_i-r_f=\alpha_i+\beta_{i,m}(r_m-r_f)+\epsilon_i
\]

is called \textbf{market model}, such \(\alpha\) is called
\textbf{Jensen's alpha}.

\hypertarget{variance-decomposition}{%
\subsubsection{Variance decomposition}\label{variance-decomposition}}

Decomposition the variance as:

\[ \var(r_i)=\overbrace{\underbrace{\beta_i\sigma_m^2}_{\text{Systematic risk}}+\underbrace{\var(\epsilon_i)}_{\text{Idiosyncratic risk}}}^{\text{total risk}} \]

The \(R^2\) is just the proportion of systematic risk

\[ R^2=\frac{\beta_i^2\sigma_m^2}{\beta_i^2\sigma_m^2+\sigma^2} \]

since
\href{https://en.wikipedia.org/wiki/Fraction_of_variance_unexplained}{Fraction
of variance unexplained}.

\hypertarget{testing-capm}{%
\subsubsection{Testing CAPM}\label{testing-capm}}

see \url{GRS.pdf}

\hypertarget{multi-factor-model}{%
\subsection{Multi-factor model}\label{multi-factor-model}}

\hypertarget{apt}{%
\subsubsection{APT}\label{apt}}

Recall in the CAPM

\[ \bm{\mathbf{r_t^e=\alpha}}+{\beta r_{m,t}^e+\nu_t} \]

Suppose now there is multi-factor with \(k\) risk factor and

\[ \bm{\mathbf{r_t=\alpha+Bf_t+\epsilon_t}} \]

Where \(\bm{\mathbf{r_t,\epsilon_t}}\) is \(n\) vector while
\(\bm{\mathbf{f_t}}\) is \(k\) vector and \(\bm{\mathbf{B}}\) is
\(n\times k\) matrix.

Taking expectation:

\[ \bm{\mathbf{r_t=\mathop{\text{E}}[r]+B(f_t-\mathop{\text{E}}[f_t])+\epsilon_t}} \]

Note CAPM is just special case of APT when \(k=1\) and
\(\bm{\mathbf{f_t}}=r_{m,t}\), which is the only factor affecting
realized return. Thta is

\[ \bm{\mathbf{r_t=\mathit{r_f}e+\beta \mathop{\text{E}}[\mathit{r_{m}-r_f}]+\beta(\mathit{r_{m,t}}-\mathop{\text{E}}[\mathit{r_{m,t}}])+\epsilon_t}} \]

Assume there is non-arbitrage, that is, if one invest \(0\) and take no
risk, then the expected return is \(0\). Formally, as \(n\to \infty\),

\[ \bm{\mathbf{\omega'\begin{bmatrix}
  \bm{\mathbf{e}}&\bm{\mathbf{B}}
\end{bmatrix}=0\implies\omega'\mathop{\text{E}}[r]=0}} \]

by Farkas lemma(Farkas 1902), that implies

\[ \begin{bmatrix}
  \bm{\mathbf{e}}&\bm{\mathbf{B}}
\end{bmatrix}\begin{bmatrix}
  \lambda_0\\\bm{\mathbf{\lambda}}
\end{bmatrix}=\mathop{\text{E}}[\bm{\mathbf{r}}] \]

for some \(\bm{\mathbf{\lambda\ge 0}}\). Under CAPM, \(\lambda_0=r_f\)
and \(\mathop{\text{E}}[\mathit{r_{m}-r_f}]=\lambda\) clearly. If one
take no risk, that is \(\bm{\mathbf{B=0}}\), s(he) get a \(\lambda_0\)
return, that implies \(\lambda_0=r_f\) immediately.

\hypertarget{when-factors-are-returns}{%
\paragraph{When factors are returns}\label{when-factors-are-returns}}

When \(\bm{\mathbf{f_t}}\) is \(\bm{\mathbf{r_t}}\), regressing it on

\[\bm{\mathbf{f_t=\mathop{\text{E}}[f]+B(f_t-\mathop{\text{E}}[f_t])+\epsilon_t}} \]

we have \(\bm{\mathbf{B=I}}\) and thus

\[ \lambda_0 \bm{\mathbf{e}}+\bm{\mathbf{\lambda}}=\mathop{\text{E}}[\bm{\mathbf{f_t}}]\implies \mathop{\text{E}}[\bm{\mathbf{f_t}}]=\bm{\mathbf{\mathit{r_f}e+\lambda }}  \]

\hypertarget{when-factors-are-excess-returns}{%
\paragraph{When factors are excess
returns}\label{when-factors-are-excess-returns}}

If \(\bm{\mathbf{f_t}}\) is excess return for some tradable assets on
time \(t\), e.g.~

\[ f_{k,t}=r_{a,t}-r_{b,t} \]

where \(r_{a,t}\) and \(r_{b,t}\) is some componets in
\(\bm{\mathbf{r_t}}\), that is

\[ f_{k,t}=\begin{bmatrix}
  0&\cdots&1&\cdots&0&\cdots&-1&\cdots&0\\
\end{bmatrix}\bm{\mathbf{r_t}} \]

where the \(a\)th componets is \(1\) while \(b\)th is \(-1\). If each
componets is excess return, we may represent \(\bm{\mathbf{f_t=Cr_t}}\)
where each row of \(\bm{\mathbf{C}}\) of such form(All zero but a \(1\)
and a \(-1\)). Recall

\[ \bm{\mathbf{r_t=\alpha+Bf_t+\epsilon_t}} \]

It follows that

\[ \bm{\mathbf{f_t=Cr_t=C\alpha+CBf_t+\epsilon}} \]

thus \(\bm{\mathbf{CB=I}}\). Recall

\[ \begin{aligned}
  r_f \bm{\mathbf{e}}+\bm{\mathbf{B\lambda}}=\mathop{\text{E}}[\bm{\mathbf{r_t}}]
\end{aligned} \]

thus

\[ \mathop{\text{E}}[\bm{\mathbf{f_t}}]=\lambda \]

as \(\bm{\mathbf{Ce=0}}\) clearly.

\hypertarget{icapm}{%
\subsubsection{ICAPM}\label{icapm}}

\hypertarget{refs}{}
\begin{cslreferences}
\leavevmode\hypertarget{ref-farkas1902theorie}{}%
Farkas, Julius. 1902. ``Theorie Der Einfachen Ungleichungen.''
\emph{Journal Für Die Reine Und Angewandte Mathematik} 1902 (124):
1--27.
\end{cslreferences}

\bibliographystyle{unsrt}
\bibliography{references.bib}


\end{document}
